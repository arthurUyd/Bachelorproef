%%=============================================================================
%% Inleiding
%%=============================================================================

\chapter{\IfLanguageName{dutch}{Inleiding}{Introduction}}%
\label{ch:inleiding}

Het onderzoeks-domein van artificiële intelligentie is in de laatste jaren op een stroomversnelling geraakt en is in korte tijd voor iedereen met internet toegang en een computer toegankelijk geworden. Deze ontwikkeling is dan ook duidelijk te zichtbaar op de bedrijfswereld zoals blijkt uit de resultaten van een  enquête over het gebruik van ICT en e-commerce in bedrijven uitgevoerd door Statbel, het Belgische statistiekbureau. Uit deze enquête blijkt dat bijna een op de zeven bedrijven gebruik maakt van artificiële intelligentie. Bij de bedrijven die meer dan 250 werknemers hebben komt dit aan een op de twee bedrijven. ~\autocite{STATBEL2023}
Deze bachelorproef kijkt naar het ontwikkelen van een chatbot die kan antwoorden op vragen die te maken hebben met losse data die op verschillende webapplicaties staan. 


\section{\IfLanguageName{dutch}{Probleemstelling}{Problem Statement}}%
\label{sec:probleemstelling}

Itineris bied software aan voor je nutsvoorzieningen zoals water of elektriciteit te beheren. Ze bieden hier ook customer service voor aan maar vaak als er een vraag gesteld wordt of er een probleem wordt vastgesteld moeten de medewerkers op meerdere webpagina's of webapplicaties de data gaan bekijken/vergelijken om vast te stellen wat er aan de hand is. Dit neemt tijd in beslag en ze kunnen soms dingen over het hoofd zien. 

\section{\IfLanguageName{dutch}{Onderzoeksvraag}{Research question}}%
\label{sec:onderzoeksvraag}

Hoe kunnen we de efficiënte ontwikkeling van een co-pilot bevorderen door het identificeren van verschillende valkuilen en technologieën met specifieke aandacht voor de vereisten van de onderneming? 

\section{\IfLanguageName{dutch}{Onderzoeksdoelstelling}{Research objective}}%
\label{sec:onderzoeksdoelstelling}

Het onderzoek moet een oog geven op de ontwikkeling van een chatbot en welke technologieën hierbij te pas komen. Een proof of concept wordt hierbij opgesteld om de keuze van de technologieën toe te lichten. 


\section{\IfLanguageName{dutch}{Opzet van deze bachelorproef}{Structure of this bachelor thesis}}%
\label{sec:opzet-bachelorproef}

% Het is gebruikelijk aan het einde van de inleiding een overzicht te
% geven van de opbouw van de rest van de tekst. Deze sectie bevat al een aanzet
% die je kan aanvullen/aanpassen in functie van je eigen tekst.

De rest van deze bachelorproef is als volgt opgebouwd:

In Hoofdstuk~\ref{ch:stand-van-zaken} wordt een overzicht gegeven van de stand van zaken binnen het onderzoeksdomein, op basis van een literatuurstudie.

In Hoofdstuk~\ref{ch:methodologie} wordt de methodologie toegelicht en worden de gebruikte onderzoekstechnieken besproken om een antwoord te kunnen formuleren op de onderzoeksvragen.

% TODO: Vul hier aan voor je eigen hoofstukken, één of twee zinnen per hoofdstuk

In Hoofdstuk~\ref{ch:conclusie}, tenslotte, wordt de conclusie gegeven en een antwoord geformuleerd op de onderzoeksvragen. Daarbij wordt ook een aanzet gegeven voor toekomstig onderzoek binnen dit domein.