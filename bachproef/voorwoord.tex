%%=============================================================================
%% Voorwoord
%%=============================================================================

\chapter*{\IfLanguageName{dutch}{Woord vooraf}{Preface}}%
\label{ch:voorwoord}

%% TODO:
%% Het voorwoord is het enige deel van de bachelorproef waar je vanuit je
%% eigen standpunt (``ik-vorm'') mag schrijven. Je kan hier bv. motiveren
%% waarom jij het onderwerp wil bespreken.
%% Vergeet ook niet te bedanken wie je geholpen/gesteund/... heeft

Met deze bachelorproef wil ik een goede manier vinden om een chatbot zinvolle antwoorden te laten geven op vragen die over bedrijfsdata gaat met het gebruik van artificiële intelligentie. Ik heb voor deze bachelorproef nog maar weinig onderzoek naar artificiële intelligentie gedaan maar toch is het altijd een onderwerp geweest dat me fascineerde. 
Ik had verschillende bedrijven gestuurd met de vraag of ze een casus hadden waar ik op kon inwerken maar toen ik antwoord kreeg met een onderwerp van Itineris wist ik al relatief snel dat dit was waar ik onderzoek naar wou doen. Daarom wil ik dus zeker mijn co-promotor, Peter Baeke, bedanken voor mij deze kans te geven en wekelijkse ondersteuning te geven waar het kon. Mijn promotor, Jan Claes, Wil ik ook hartelijk bedanken voor altijd snelle feedback te geven en te helpen met vragen die ik had over het verloop van de bachelorproef. Voor mijn meer technische vragen kon ik ook altijd terecht bij Maarten Glas en Tom Uvin die me ook doorheen het verloop feedback hebben gegeven op mijn werk. 
Ik wens u veel leesplezier en hoop dat u iets nuttig kan halen uit deze bachelorproef!


