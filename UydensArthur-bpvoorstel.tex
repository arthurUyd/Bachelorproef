%==============================================================================
% Sjabloon onderzoeksvoorstel bachproef
%==============================================================================
% Gebaseerd op document class `hogent-article'
% zie <https://github.com/HoGentTIN/latex-hogent-article>

% Voor een voorstel in het Engels: voeg de documentclass-optie [english] toe.
% Let op: kan enkel na toestemming van de bachelorproefcoördinator!
\documentclass{hogent-article}

% Invoegen bibliografiebestand
\addbibresource{voorstel.bib}

% Informatie over de opleiding, het vak en soort opdracht
\studyprogramme{Professionele bachelor toegepaste informatica}
\course{Bachelorproef}
\assignmenttype{Onderzoeksvoorstel}
% Voor een voorstel in het Engels, haal de volgende 3 regels uit commentaar
% \studyprogramme{Bachelor of applied information technology}
% \course{Bachelor thesis}
% \assignmenttype{Research proposal}

\academicyear{2023-2024} % TODO: pas het academiejaar aan

% TODO: Werktitel
\title{Efficiënte Ontwikkeling van Schaalbare Enterprise Webapplicaties: Een Onderzoek naar Best Practices en Technologieën}

% TODO: Studentnaam en emailadres invullen
\author{Arthur Uydens }
\email{arthur.uydens@student.hogent.be}

% TODO: Medestudent
% Gaat het om een bachelorproef in samenwerking met een student in een andere
% opleiding? Geef dan de naam en emailadres hier
% \author{Yasmine Alaoui (naam opleiding)}
% \email{yasmine.alaoui@student.hogent.be}

% TODO: Geef de co-promotor op
\supervisor[Co-promotor]{S. Beekman (Synalco, \href{mailto:sigrid.beekman@synalco.be}{sigrid.beekman@synalco.be})}

% Binnen welke specialisatierichting uit 3TI situeert dit onderzoek zich?
% Kies uit deze lijst:
%
% - Mobile \& Enterprise development
% - AI \& Data Engineering
% - Functional \& Business Analysis
% - System \& Network Administrator
% - Mainframe Expert
% - Als het onderzoek niet past binnen een van deze domeinen specifieer je deze
%   zelf
%
\specialisation{Mobile \& Enterprise development}
\keywords{Enterprise Webapplicaties, Schaalbaarheid, Best Practices, Technologieën, Onderzoek}

\begin{document}

\begin{samenvatting}
  Dit onderzoeksvoorstel richt zich op de efficiënte ontwikkeling van schaalbare enterprise webapplicaties, een groeiend en belangrijk aspect in de snel evoluerende zakelijke omgeving. Na een voorstelling van het thema en een kadering van de relevantie van krachtige webapplicaties voor ondernemingen, identificeert het de centrale probleemstelling met de noodzaak om best practices en geschikte technologieën te vinden voor het ontwikkelen van dergelijke applicaties. De onderzoeksvraag richt zich op het identificeren van vereisten voor schaalbare zulke webapplicaties en het evalueren van de toepasbaarheid van bestaande ontwikkelingspraktijken. De doelstellingen omvatten het vergelijken van bestaande web development frameworks, het ontwikkelen van een prototype en het analyseren van dit prototype. De voorgestelde methodologie omvat een literatuurstudie, casestudy's, frameworkvergelijking, prototype-ontwikkeling en gebruikerstests. De verwachte resultaten omvatten een overzicht van vereisten, identificatie van best practices, een frameworkvergelijking en een werkend prototype. De meerwaarde van dit onderzoek ligt in het bieden van praktische inzichten voor ontwikkelaars en bedrijven die streven naar schaalbare, efficiënte webapplicaties, wat de doelgroep zal helpen bij het verbeteren van hun ontwikkelingsprocessen en het kiezen van geschikte technologieën voor enterprise web development.

\section*{Introductie}
De behoefte aan krachtige, schaalbare webapplicaties binnen ondernemingen groeit snel. Dit onderzoeksvoorstel streeft naar het identificeren en evalueren van best practices en technologieën die de efficiënte ontwikkeling van schaalbare enterprise webapplicaties helpen bevorderen.

De doelgroep van deze bachelorproef zijn bedrijven die webapplicaties voor andere ondernemingen ontwikkelen. Hier zijn vele voorbeelden van zoals Delaware, HiveCPQ, etc... Deze bedrijven hebben vaak te maken met een grote hoeveelheid gebruikers en data. Het is dus belangrijk dat de applicaties die zij ontwikkelen schaalbaar zijn. Dit betekent dat de applicaties in staat moeten zijn om te groeien en zich aan te passen aan de veranderende behoeften van de gebruikers. Daarnaast moeten de applicaties ook goed presteren en veilig zijn.

De onderzoeksvraag luidt dus: \textbf{
  Hoe kunnen we de efficiënte ontwikkeling van schaalbare enterprise webapplicaties bevorderen door de identificatie en evaluatie van best practices en technologieën, met specifieke aandacht voor de vereisten van ondernemingen?
}

\section*{Doelstellingen}
\begin{enumerate}
  \item Identificeren van de belangrijkste vereisten voor enterprise webapplicaties, met de nadruk op schaalbaarheid, prestaties en beveiliging.
    \begin{itemize}
      \item Welke vereisten zijn het belangrijkst voor de schaalbaarheid van een enterprise webapplicatie?
      \item Welke vereisten zijn het belangrijkst voor de prestaties van een enterprise webapplicatie?
      \item Welke vereisten zijn het belangrijkst voor de beveiliging van een enterprise webapplicatie?
    \end{itemize}
  \item Onderzoeken van best practices in de ontwikkeling van schaalbare webapplicaties en hun toepasbaarheid in een ondernemingsomgeving.
    \begin{itemize}
      \item Wat zijn de best practices voor de ontwikkeling van schaalbare webapplicaties?
      \item Hoe kunnen deze best practices worden toegepast in een ondernemingsomgeving?
      \item Waarom zijn deze best practices belangrijk voor ondernemingen?
    \end{itemize}
  \item Vergelijken van verschillende web development frameworks en technologieën op het gebied van schaalbaarheid, prestaties en onderhoud.
    \begin{itemize}
      \item Wat zijn de meest courante web development frameworks en technologieën?
      \item Waarom worden deze frameworks en technologieën zo veel gebruikt?
    \end{itemize}
  \item Ontwikkelen van een prototype van een schaalbare enterprise webapplicatie en evalueren van de gemaakte keuzes in de ontwikkeling.
\end{enumerate}

\section*{Stand van zaken}
Over heel de wereld zijn er al veel bedrijven die schaalbare entreprise webapplicaties maken en dit op een sterke maar ook steeds verschillende manier in stand brengen.

HiveCPQ is de nieuwe leider op de belgische markt in het aanbieden van CPQ-oplossingen aan enterprise-fabrikanten. Je hebt de term CPQ ongetwijfeld al zien voorbijkomen, zeker als je in de maakindustrie werkt. Maar wat is CPQ of Configure-Price-Quote software? CPQ-software helpt bedrijven om hun producten en diensten te configureren, te prijzen en te verkopen. Het is een software die de verkoop van complexe producten en diensten automatiseert en versnelt. CPQ-software is een must voor bedrijven die hun verkoopproces willen optimaliseren en hun omzet willen verhogen. 

Daarnaast hebben we dan ook Delaware, een bedrijf dat begon als een onafhankelijk partnerschap en zich nu heeft uitgewerkt tot een wereldwijde aanwezigheid met meer dan 4600 medewerkers. Delaware is een bedrijf dat geavanceerde ICT-oplossingen en -diensten levert en onze klanten begeleidt bij hun zakelijke en digitale transformaties naar de digitale toekomst.

\section*{Methodologie}
\begin{enumerate}
  \item Literatuurstudie: 
  \begin{itemize}
    \item Onderzoek naar bestaande literatuur en documentatie met betrekking tot de ontwikkeling van schaalbare enterprise webapplicaties.
  \end{itemize}
  \item Casestudy's: 
  \begin{itemize}
    \item Analyse van bestaande enterprise webapplicaties om inzicht te krijgen in de gehanteerde methodologieën en technologieën.
  \end{itemize}
  \item Frameworkvergelijking:
  \begin{itemize}
    \item vergelijken van populaire web development frameworks op basis van criteria zoals schaalbaarheid, prestaties en onderhoud.
  \end{itemize}
  \item Prototype-ontwikkeling: 
  \begin{itemize}
    \item Ontwikkeling van een prototype van een schaalbare enterprise webapplicatie, gebruikmakend van de geïdentificeerde best practices en technologieën.
  \end{itemize}
  \item Gebruikerstest:
  \begin{itemize}
    \item Verzamelen van feedback van gebruikers en ontwikkelaars over de ervaringen met het ontwikkelde prototype.
  \end{itemize}
\end{enumerate}

\section*{Verwachte resultaten}
\begin{enumerate}
  \item Een overzicht van de belangrijkste vereisten voor schaalbare enterprise webapplicaties.
  \item Identificatie van best practices in de ontwikkeling van schaalbare webapplicaties en hun relevantie voor ondernemingen.
  \item Een vergelijkende analyse van web development frameworks en technologieën op het gebied van schaalbaarheid en prestaties.
  \item Een werkend prototype van een schaalbare enterprise webapplicatie.
\end{enumerate}

\end{abstract}

\tableofcontents

% De hoofdtekst van het voorstel zit in een apart bestand, zodat het makkelijk
% kan opgenomen worden in de bijlagen van de bachelorproef zelf.
\input{voorstel-inhoud}

\printbibliography[heading=bibintoc]

\end{document}