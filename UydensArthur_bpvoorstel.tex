%!TEX program = xelatex

\documentclass{hogent-article}

\addbibresource{voorstel.bib}

\studyprogramme{Professionele bachelor toegepaste informatica}
\course{Bachelorproef}
\assignmenttype{Onderzoeksvoorstel}


\academicyear{2023-2024} 


\title{De ontwikkeling van een co-pilot gebaseerd op een MS Dynamics 365 platform}

\author{Arthur Uydens }
\email{arthur.uydens@student.hogent.be}


\supervisor[Co-promotor]{}

\specialisation{Mobile \& Enterprise development}
\keywords{co-pilot, MS Dynamics 365, AI , chatbot}

\begin{document}

\section{Samenvatting}
Dit onderzoeksvoorstel richt zich op het ontwikkelen van een co-pilot op een platform dat gebruik maakt van het MS-Dynamics 365 ERP-pakket.De opdracht is in naam van Itineris waarvan Peter Baeke, R&D Manager Business Unit North America, de rol van co-promotor opneemt en me inhoudelijk zal bijstaan. De co-pilot zou de medewerkers moeten ondersteunen in hun business handelingen en de werkdruk moeten verlagen. Er zal naar zowel het gebruik als naar het correct trainen van de co-pilot gekeken worden om uiteindelijk tot een werkende en efficiënte co-pilot te komen. Er zal onderzocht worden naar de beste manieren waarop de co-pilot gebruikt zal kunnen worden, waar er eventuele valkuilen zitten, hoe de co-pilot getraind kan worden en wat de beste manier is om deze training te voltooien. Met al deze informatie zal er dan een product gemaakt worden om deze opdracht succesvol te voltooien. Met dit eindproduct zullen er verschillende scenario's uitgevoerd worden waarop er dan aan de hand van verschillende criteria beslist wordt of het scenario succesvol is uitgevoerd.

\section{Introductie}
Aritificiële Intelligentie is de wetenschap van de toekomst. Indien je als bedrijf niet op inzet in het jaar 2024 zal je bedrijf een achterstand opwerken op de competitie. Dit onderzoeksvoorstel streeft naar praktische ondersteuning en info te geven over hoe een co-pilot systeem gebruikt kan worden binnen de bestaande applicatie die Itineris aanbied.  

De doelgroep van deze bachelorproef zijn bedrijven die webapplicaties voor andere ondernemingen ontwikkelen. Hier zijn vele voorbeelden van zoals Delaware, HiveCPQ, etc... Deze bedrijven hebben vaak te maken met een grote hoeveelheid gebruikers en data. Het is dus belangrijk dat de applicaties die zij ontwikkelen schaalbaar zijn. Dit betekent dat de applicaties in staat moeten zijn om te groeien en zich aan te passen aan de veranderende behoeften van de gebruikers. Daarnaast moeten de applicaties ook goed presteren en veilig zijn.

De onderzoeksvraag luidt dus: \textbf{
  onderzoeksvraag: Hoe kunnen we de efficiënte ontwikkeling van een co-pilot bevorderen door het identificeren van verschillende valkuilen en trainingsmethoden met specifieke aandacht voor de vereisten van de onderneming? 
}

\section{Doelstellingen}
\begin{enumerate}
  \item Identificeren van de belangrijkste vereisten voor enterprise webapplicaties, met de nadruk op schaalbaarheid, prestaties en beveiliging.
    \begin{itemize}
      \item Welke vereisten zijn het belangrijkst voor de schaalbaarheid van een enterprise webapplicatie?
      \item Welke vereisten zijn het belangrijkst voor de prestaties van een enterprise webapplicatie?
      \item Welke vereisten zijn het belangrijkst voor de beveiliging van een enterprise webapplicatie?
    \end{itemize}
  \item Onderzoeken van best practices in de ontwikkeling van schaalbare webapplicaties en hun toepasbaarheid in een ondernemingsomgeving.
    \begin{itemize}
      \item Wat zijn de best practices voor de ontwikkeling van schaalbare webapplicaties?
      \item Hoe kunnen deze best practices worden toegepast in een ondernemingsomgeving?
      \item Waarom zijn deze best practices belangrijk voor ondernemingen?
    \end{itemize}
  \item Vergelijken van verschillende web development frameworks en technologieën op het gebied van schaalbaarheid, prestaties en onderhoud.
    \begin{itemize}
      \item Wat zijn de meest courante web development frameworks en technologieën?
      \item Waarom worden deze frameworks en technologieën zo veel gebruikt?
    \end{itemize}
  \item Ontwikkelen van een prototype van een schaalbare enterprise webapplicatie en evalueren van de gemaakte keuzes in de ontwikkeling.
\end{enumerate}

\section{Stand van zaken}
Over heel de wereld zijn er al veel bedrijven die schaalbare entreprise webapplicaties maken en dit op een sterke maar ook steeds verschillende manier in stand brengen.

HiveCPQ is de nieuwe leider op de belgische markt in het aanbieden van CPQ-oplossingen aan enterprise-fabrikanten. Je hebt de term CPQ ongetwijfeld al zien voorbijkomen, zeker als je in de maakindustrie werkt. Maar wat is CPQ of Configure-Price-Quote software? CPQ-software helpt bedrijven om hun producten en diensten te configureren, te prijzen en te verkopen. Het is een software die de verkoop van complexe producten en diensten automatiseert en versnelt. CPQ-software is een must voor bedrijven die hun verkoopproces willen optimaliseren en hun omzet willen verhogen. 

Daarnaast hebben we dan ook Delaware, een bedrijf dat begon als een onafhankelijk partnerschap en zich nu heeft uitgewerkt tot een wereldwijde aanwezigheid met meer dan 4600 medewerkers. Delaware is een bedrijf dat geavanceerde ICT-oplossingen en -diensten levert en onze klanten begeleidt bij hun zakelijke en digitale transformaties naar de digitale toekomst.

\section{Methodologie}
\begin{enumerate}
  \item Literatuurstudie: 
  \begin{itemize}
    \item Onderzoek naar bestaande literatuur en documentatie met betrekking tot de ontwikkeling van schaalbare enterprise webapplicaties.
  \end{itemize}
  \item Casestudy's: 
  \begin{itemize}
    \item Analyse van bestaande enterprise webapplicaties om inzicht te krijgen in de gehanteerde methodologieën en technologieën.
  \end{itemize}
  \item Frameworkvergelijking:
  \begin{itemize}
    \item vergelijken van populaire web development frameworks op basis van criteria zoals schaalbaarheid, prestaties en onderhoud.
  \end{itemize}
  \item Prototype-ontwikkeling: 
  \begin{itemize}
    \item Ontwikkeling van een prototype van een schaalbare enterprise webapplicatie, gebruikmakend van de geïdentificeerde best practices en technologieën.
  \end{itemize}
  \item Gebruikerstest:
  \begin{itemize}
    \item Verzamelen van feedback van gebruikers en ontwikkelaars over de ervaringen met het ontwikkelde prototype.
  \end{itemize}
\end{enumerate}

\section{Verwachte resultaten}
\begin{enumerate}
  \item Een overzicht van de belangrijkste vereisten voor schaalbare enterprise webapplicaties.
  \item Identificatie van best practices in de ontwikkeling van schaalbare webapplicaties en hun relevantie voor ondernemingen.
  \item Een vergelijkende analyse van web development frameworks en technologieën op het gebied van schaalbaarheid en prestaties.
  \item Een werkend prototype van een schaalbare enterprise webapplicatie.
\end{enumerate}



\tableofcontents


\end{document}